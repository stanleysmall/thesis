\documentclass{article}
\usepackage[utf8]{inputenc}

\usepackage[american]{babel}
\usepackage{csquotes}
\usepackage[style=apa, natbib]{biblatex}
\DeclareLanguageMapping{american}{american-apa}
% add the bibliography
\addbibresource{readinglist.bib}

\title{Honors Reading List}
\author{Stanley C. Small}
\date{2018 - 2019}

\DeclareLabelname[movie]{
      \field{director}
      \field{producer}
    }

\begin{document}

\begin{titlepage}
\maketitle
\tableofcontents
\thispagestyle{empty}
\end{titlepage}

\newpage
\section{Introduction}
This work discusses twelve titles contributing to my intellectual development throughout university and serves to provide an "entr\'{e}e" into the works for the committee. 

\section{My Thoughts}

\subsection{A Mind for Numbers}
In \textit{A Mind for Numbers}, Dr. Barbera Oakley describes methods to learn science and mathematics effectively. Oakley explains why certain studying techniques allow one to retain and make new connections easier. Rarely do public schools or universities teach one the underlying principles of learning. The text provides aspiring lifelong learners with tools to study and comprehend complicated material.  

Oakley suggests completing the most difficult tasks first, because the brain often continues working on problems while individuals complete less mentally stressful activities. Taking breaks remains essential to the success of a student. By learning how to learn, one can live a better life. 

\subsection{The Inner Game of Tennis}
Tennis, a sport of both mental and physical endurance, can arouse frustration in even the most experienced players. Timothy Gallwey, author of \textit{The Inner Game of Tennis} believes individuals who learn to trust their bodies and themselves play their best game. The work not only applies to tennis, but a variety of endeavors requiring peak mental performance. One could generalize lessons of the work to frustrations of life. Gallwey argues battles within the mind of the player outweigh those between him and his opponent when trying to play the game. 

The book teaches one to avoid perfectionistic tendencies doing him a disservice. Most importantly, the book teaches individuals to notice errors and move on, rather than to dwell or ruminate on past mistakes. One can learn to accept flaws as part of his game and his life without letting them define him. Everyone learns to accept their mistakes at some point in life. Learning, adjusting, and moving on from mistakes remains the best strategy for achieving success. By learning to focus on the present moment, one can often achieve better performance than when dwelling on the past or future. 

Gallwey maintains the ability to concentrate on the present allows one to focus his mind and to live calmly. By dedicating one's self towards a specific task at a specific moment in time, he can find peace. 

\subsection{Trying Not to Try}
\textit{Trying Not to Try} also describes methods to encourage effortless action, referred to the text as \emph{wu wei}. Eastern philosophy sits at the heart of Edward Slingerland's text. He presents fours levels of understanding: unconscious incompetence, conscious incompetence, conscious competence, and unconscious competence. I have the utmost admiration for artists who make their work appear effortless and strive to attain such a level of expertise. 

Slingerland also explores \emph{de}, loosely translated as "charismatic power", which draws people near and makes an individual seem trustworthy. The spontaneous and unconscious nature of actions exercised in  \emph{wu wei} allow one to have great influence over others. The text attempts to explain how one can capture an audience or lead a crowd. I have attempted to take the concept of effortless action to heart in an effort to maintain charisma and sanity in a chaotic universe. 

One might believe achieving one's goals depends on careful analysis and reasoning while constantly striving. However, the constant effort detracts from many human qualities, such as happiness and spontaneity, which one best pursues indirectly. Many people wonder how to achieve happiness, and Eastern thought encourages individuals to seek peace through inaction. 
\subsection{I Ching}
The \textit{I Ching}, a divination text, provides additional insight into Eastern thought and philosophy and  resembles a spiritual text more than any other presented here. Brian Browne Walker translates the original Chinese hexagrams in a beautiful poetic style. Classically one consults the \textit{I Ching} by tossing three coins and turning to an indicated page. The text seeks to provide guidance in times of moral ambiguity, with influences from Confucianism, Taoism and Buddhism. 

The text can provide consolation for those feeling hopeless or lost. The \textit{I Ching} often recommends personal retreat and quiet reflection. Some situations require immediate and reactive measures, but most struggles of the human condition can soften with time and humble acceptance. 

For individuals suffering from anxiety, the \textit{I Ching} provides a sense of reassurance and comfort in troubling times and encourages the reader to enter a contemplative state of mind. 

\section{My Career}

\subsection{Algorithms to Live By}
\textit{Algorithms to Live By} explains how one can apply common algorithms taught in a typical computer science program to real-world scenarios. The text explains how one can find inspiration for and from computation in unlikely places. For example, the text details the idea of "optimal-stopping" with regards to finding the best apartment or picking the best life partner. The text also provides an explanation for why the elderly prefer routine and children prefer novelty, noting the chances of finding better experiences as one ages significantly diminish. 

Brian Christian and Tom Griffiths provide algorithms with remarkable applications for daily human life. Readers walk away with 
knowledge to live a more productive and better organized life, as computing algorithms remain a surprisingly useful way to embrace complexity. 

\subsection{Black Mirror}
\textit{Black Mirror}, named after the appearance of a dark screen, examines unanticipated consequences of new technologies. Similar to \textit{The Twilight Zone}, the series portrays a scene not unlike modern society, but often has a dark twist. The series appeals to those with an affinity for technology and an optimism for its prospects.

Each episode has a capacity to leave the viewer speechless. The show allows one to consider possible impacts of future technology and the responsibilities of those developing such technologies. The series addresses a multitude of ethical and moral complications surrounding new technologies. 

\subsection{The Signal and the Noise}
Nate Silver, founder and editor in chief of FiveThirtyEight, wrote \textit{The Signal and the Noise} when optimism surrounding predictive technologies peaked. His work describes how information and analysis can provide insights for some of the toughest problems. The text provides a number of success stories written for an audience optimistic about the future of technology. 

The book explains how predictions can help to uncover mysteries or to solve problems, as long as one remains aware of his biases. By learning to identify one's own bias and the bias inherent in various systems, he makes better and more accurate predications. 

\subsection{Disrupted}
Dan Lyons, a former columnist, writes about his adventures working for Silicon Valley startup Hubspot in the comedy \textit{Disrupted}. Lyons later went on to be a writer for the HBO television series Silicon Valley, an American comedy series. The novel painted a vision of silicon valley in an exciting light, with startups and entrepreneurs and venture capital. The text ultimately focused on the absurdity of startup culture, where everyone wishes to acquire ludicrous wealth. 

Lyons captures the absurdity of startup culture, and the enormous pressure for growth. \textit{Disrupted} provides a gripping narrative in a humorous manner. Any optimist would greatly enjoy his tale, but would also take caution for the future of technology and his future careers. 

\section{My Worldview}
\subsection{What Every BODY is Saying}
Joe Navarro, a former FBI agent, explains how one's body communicates when he remains silent. By learning to read the nonverbal communication of others, individuals can learn to understand the hidden intentions of others. Identifying involuntary conditions of the human body also allows one to live as an empathetic and understanding person. Understanding the unconscious movements of the human body allows for a more complete comprehension of the human condition and an elevated respect for nature. 

Identifying the unconscious movements or behaviors of one's own body can help him notice and better understand his current emotional and physiological state. Moreover, when one can more clearly identify his emotions along with the feelings and anxieties of others, he can live a more compassionate life. 

\subsection{Mr. Nobody}
\textit{Mr. Nobody} explores the consequences of one's decisions. The film presents multiple timelines for Nemo Nobody's life, the last mortal alive after the human race has achieved quasi-immortality, in a nonlinear narrative. Viewers witness the protagonist in multiple universes with different wives and different careers. The film provides a commentary on the fragility of life and the consequences which result from seemingly insignificant actions. 

After viewing the film, one may contemplate the sheer complexity of human life and the complex interactions between individuals occurring at every moment. One might begin to consider his actions more carefully, or start to view others in a more compassionate light. \textit{Mr. Nobody} captures the beautiful air of uncertainty which plagues every living indivudal. 

\subsection{The Lobster}
\textit{The Lobster} explores the absurdity of love. In the film's setting, single people must find a romantic partner within 45 days to avoid being turned into an animal. The film provides a commentary on the pressures of society to find a lifelong companion and the challenges of both romantic involvement and single life. 

The film portrays an elusive yet beautiful conceptualization of love and the stresses one willingly undertakes to achieve such. The viewer questions to himself if love truly exists, and how he can know. 
\subsection{BoJack Horseman}
If anyone lives long enough, he will question his existence. \textit{BoJack Horseman}, an animated series, captures the life of a character sharing the same name suffering from existential nihilism. Reviewers laud the series for a realistic depiction of an individual suffering from depression, trauma, addiction, and self-destructive behavior. Any individual suffering from similar conditions might find solace through a sense of shared hardship. 

Individuals identifying with the protagonist might come to empathize with his failed efforts to sustain relationships or a sense of happiness. The series offers a unique perspective into the life of one (Horse)man all too familiar with humanities search for meaning and fulfillment. While the series does not provide resolution to the protagonist's troubles, many important discussions regarding life, happiness, and fulfillment transpire. 

\newpage
\nocite{*}
\printbibliography
\end{document}


